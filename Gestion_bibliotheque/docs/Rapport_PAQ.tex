\documentclass[12pt,a4paper]{article}
\usepackage[utf8]{inputenc}
\usepackage[french]{babel}
\usepackage{graphicx}
\usepackage{xcolor}
\usepackage{hyperref}
\usepackage{listings}
\usepackage{geometry}
\usepackage{fancyhdr}
\usepackage{titlesec}

\geometry{left=2.5cm,right=2.5cm,top=2.5cm,bottom=2.5cm}

% Configuration des couleurs
\definecolor{codegreen}{rgb}{0,0.6,0}
\definecolor{codegray}{rgb}{0.5,0.5,0.5}
\definecolor{codepurple}{rgb}{0.58,0,0.82}
\definecolor{backcolour}{rgb}{0.95,0.95,0.92}

% Configuration du style de code
\lstdefinestyle{mystyle}{
    backgroundcolor=\color{backcolour},
    commentstyle=\color{codegreen},
    keywordstyle=\color{blue},
    numberstyle=\tiny\color{codegray},
    stringstyle=\color{codepurple},
    basicstyle=\ttfamily\footnotesize,
    breakatwhitespace=false,
    breaklines=true,
    captionpos=b,
    keepspaces=true,
    numbers=left,
    numbersep=5pt,
    showspaces=false,
    showstringspaces=false,
    showtabs=false,
    tabsize=2
}
\lstset{style=mystyle}

% En-tête et pied de page
\pagestyle{fancy}
\fancyhf{}
\fancyhead[L]{PAQ - Gestion Bibliothèque}
\fancyhead[R]{\thepage}
\fancyfoot[C]{Plan d'Assurance Qualité}

\begin{document}

% Page de garde
\begin{titlepage}
    \centering
    \vspace*{2cm}
    
    {\Huge\bfseries Plan d'Assurance Qualité (PAQ)\par}
    \vspace{0.5cm}
    {\Large Application de Gestion de Bibliothèque\par}
    
    \vspace{2cm}
    
    {\Large\textbf{Technologies utilisées}\par}
    \vspace{0.5cm}
    {\large Spring Boot 4.0.2 | Spring MVC | JPA/Hibernate | MySQL\par}
    {\large JUnit 5 | Mockito | MockMvc | Thymeleaf\par}
    
    \vspace{3cm}
    
    {\large Développé par : \textbf{USER1}\par}
    \vspace{0.5cm}
    {\large Établissement : \textbf{FSTM - ILISI}\par}
    
    \vfill
    
    {\large Date : 17 Février 2026\par}
\end{titlepage}

\newpage
\tableofcontents
\newpage

% ====================== PARTIE 1 ======================
\section{Partie 1 : Cadre et Référentiels}

\subsection{Périmètre du PAQ}

Le PAQ couvre le développement complet de l'application web \textbf{"Gestion de Bibliothèque"}. 

\subsubsection{Fonctionnalités incluses}
\begin{itemize}
    \item \textbf{Backend} : API Spring MVC avec fonctionnalités CRUD complètes
    \item \textbf{Frontend} : Interface web avec Thymeleaf (templates HTML)
    \item \textbf{Base de données} : Persistance MySQL avec JPA/Hibernate
    \item \textbf{Fonctionnalités} :
    \begin{itemize}
        \item Affichage et recherche de livres avec pagination
        \item Ajout de nouveaux livres
        \item Modification des informations d'un livre
        \item Suppression de livres
    \end{itemize}
\end{itemize}

\subsubsection{Technologies et versions}
\begin{itemize}
    \item Spring Boot : 4.0.2
    \item Java : 21.0.9 (LTS)
    \item Maven : Gestion des dépendances
    \item MySQL : Base de données
    \item Bootstrap : 5.3.8 (Interface utilisateur)
\end{itemize}

\subsection{Organisation du projet et responsabilités}

\textbf{Développeur Principal \& Responsable Qualité} : USER1

\textbf{Responsabilités} :
\begin{itemize}
    \item Architecture Spring Boot et conception MVC
    \item Développement des fonctionnalités CRUD
    \item Écriture des tests unitaires et d'intégration
    \item Vérification de la qualité du code
    \item Respect des délais de livraison
\end{itemize}

\textit{Note : Projet réalisé individuellement, sans binôme.}

\subsection{Démarche d'assurance qualité}

\textbf{Approche préventive} adoptée :

\begin{enumerate}
    \item \textbf{Avant chaque commit} :
    \begin{itemize}
        \item Compilation sans erreur vérifiée
        \item Tests unitaires exécutés et validés
        \item Code formaté selon les conventions Java
    \end{itemize}
    
    \item \textbf{Contrôle continu} :
    \begin{itemize}
        \item Utilisation de Git pour le versionnement
        \item GitHub Actions pour l'intégration continue (prévu)
        \item Revue de code avant fusion des fonctionnalités
    \end{itemize}
    
    \item \textbf{Documentation} :
    \begin{itemize}
        \item Commentaires Javadoc sur les méthodes complexes
        \item README avec instructions de déploiement
    \end{itemize}
\end{enumerate}

\subsection{Référentiels et Standards de codage}

\subsubsection{Conventions de nommage}
\begin{itemize}
    \item \textbf{Variables} : camelCase (\texttt{titreLivre}, \texttt{quantiteStock})
    \item \textbf{Classes} : PascalCase (\texttt{LivreController}, \texttt{LivreRepository})
    \item \textbf{Constantes} : UPPER\_SNAKE\_CASE (\texttt{MAX\_PAGE\_SIZE})
    \item \textbf{Packages} : lowercase (\texttt{fstm.ilisi.gestion\_bibliotheque})
\end{itemize}

\subsubsection{Architecture Spring Boot}
Respect strict des couches MVC :

\begin{itemize}
    \item \textbf{Entity} : Modèle de données JPA
    \begin{lstlisting}[language=Java]
@Entity
public class Livre {
    @Id @GeneratedValue
    private Long id;
    private String titre;
    private String auteur;
    private int Quantite;
}
    \end{lstlisting}
    
    \item \textbf{Repository} : Interface JpaRepository pour accès aux données
    \begin{lstlisting}[language=Java]
public interface LivreRepository 
    extends JpaRepository<Livre, Long> {
    Page<Livre> findByTitreContainsIgnoreCase(...);
}
    \end{lstlisting}
    
    \item \textbf{Controller} : Points d'entrée MVC (GET/POST)
    \begin{lstlisting}[language=Java]
@Controller
public class LivreController {
    @GetMapping("/index")
    public String index(Model model) { ... }
}
    \end{lstlisting}
\end{itemize}

% ====================== PARTIE 2 ======================
\newpage
\section{Partie 2 : Tests et Contrôle}

\subsection{Procédures qualité}

\subsubsection{Gestion des versions}
\begin{itemize}
    \item \textbf{Outil} : Git
    \item \textbf{Stratégie de branches} :
    \begin{itemize}
        \item \texttt{main} : Branche principale stable
        \item \texttt{feature/nom-fonctionnalite} : Développement des nouvelles fonctionnalités
        \item Fusion après validation des tests
    \end{itemize}
\end{itemize}

\subsection{Stratégie de tests}

\subsubsection{Tests Unitaires : JUnit 5 + Mockito}

\textbf{Objectif} : Tester chaque méthode du contrôleur en isolant les dépendances.

\textbf{Technologies} :
\begin{itemize}
    \item \textbf{JUnit 5 (Jupiter)} : Framework de tests moderne
    \item \textbf{Mockito} : Mock des dépendances (Repository)
    \item \textbf{MockMvc} : Simulation de requêtes HTTP
\end{itemize}

\textbf{Exemple de test unitaire} :
\begin{lstlisting}[language=Java]
@Test
@DisplayName("Test deleteLivre")
void testDeleteLivre() {
    // ARRANGE
    doNothing().when(livreRepository).deleteById(1L);
    
    // ACT
    String redirect = livreController.deleteLivre(1L, 0, "");
    
    // ASSERT
    assertEquals("redirect:/index?page=0&search=", redirect);
    verify(livreRepository).deleteById(1L);
}
\end{lstlisting}

\textbf{Pattern AAA (Arrange-Act-Assert)} systématiquement appliqué.

\subsubsection{Tests d'Intégration}

\textbf{Objectif} : Vérifier le contexte Spring Boot complet avec base de données.

\begin{lstlisting}[language=Java]
@SpringBootTest
class GestionBibliothequeApplicationTests {
    @Test
    void contextLoads() {
        // Vérifie que le contexte Spring démarre
    }
}
\end{lstlisting}

\subsubsection{Couverture des tests}

\begin{table}[h]
\centering
\begin{tabular}{|l|c|c|}
\hline
\textbf{Méthode testée} & \textbf{Type} & \textbf{Statut} \\
\hline
\texttt{index()} & Unitaire + MockMvc & \textcolor{green}{✓} \\
\texttt{deleteLivre()} & Unitaire & \textcolor{green}{✓} \\
\texttt{showEditForm()} & Unitaire & \textcolor{green}{✓} \\
\texttt{saveLivre()} & Unitaire & \textcolor{green}{✓} \\
\texttt{showAddForm()} & Unitaire & \textcolor{green}{✓} \\
\texttt{addLivre()} & Unitaire & \textcolor{green}{✓} \\
\texttt{home()} & Unitaire & \textcolor{green}{✓} \\
Contexte Spring & Intégration & \textcolor{green}{✓} \\
\hline
\end{tabular}
\caption{Couverture des tests - LivreController}
\end{table}

\subsection{Résultats des tests}

\subsubsection{Exécution Maven}

Commande exécutée : \texttt{mvn test}

\begin{verbatim}
[INFO] Tests run: 9, Failures: 0, Errors: 0, Skipped: 0
[INFO] BUILD SUCCESS
[INFO] Total time:  14.769 s
[INFO] Finished at: 2026-02-17T11:27:18Z
\end{verbatim}

\textbf{Détail des résultats} :
\begin{itemize}
    \item \textbf{LivreControllerTest} : 8 tests - Temps : 3.223s
    \item \textbf{GestionBibliothequeApplicationTests} : 1 test - Temps : 5.741s
    \item \textbf{Taux de réussite} : 100\% (9/9)
\end{itemize}

\subsubsection{Logs de test significatifs}
\begin{itemize}
    \item Spring Boot 4.0.2 démarré avec succès
    \item Connexion MySQL établie (HikariCP)
    \item Hibernate ORM 7.2.1 initialisé
    \item MockMvc configuré pour tests HTTP
    \item Repository JPA scanné et chargé
\end{itemize}

\subsection{Gestion des anomalies}

\textbf{Procédure de suivi} :
\begin{enumerate}
    \item \textbf{Détection} : Via tests automatiques ou test manuel
    \item \textbf{Classification} :
    \begin{itemize}
        \item \textcolor{red}{Critique} : Bloquant (perte de données, crash application)
        \item \textcolor{orange}{Majeur} : Fonctionnalité non opérationnelle
        \item \textcolor{yellow}{Mineur} : Bug cosmétique ou ergonomique
    \end{itemize}
    \item \textbf{Traçabilité} : Utilisation des Issues GitHub
    \item \textbf{Résolution} : Correction avant livraison finale pour bugs critiques/majeurs
\end{enumerate}

\textbf{Exemple d'anomalie détectée et corrigée} :
\begin{itemize}
    \item \textit{Problème} : Navbar Bootstrap non affichée (classes Bootstrap 3 avec version 5)
    \item \textit{Classification} : Majeur
    \item \textit{Résolution} : Migration vers classes Bootstrap 5 + utilisation de fragments Thymeleaf
\end{itemize}

\subsection{Audits et revues qualité}

\subsubsection{Revue de code}

Vérifications systématiques avant fusion :
\begin{itemize}
    \item \textbf{Code mort} : Suppression du code inutilisé
    \item \textbf{Gestion des exceptions} : Pas de \texttt{try/catch} vides
    \item \textbf{Commentaires} : Javadoc sur méthodes complexes
    \item \textbf{Conventions} : Respect du style Java
    \item \textbf{Sécurité} : Validation des entrées utilisateur
\end{itemize}

\subsubsection{Métriques qualité}

\begin{table}[h]
\centering
\begin{tabular}{|l|c|}
\hline
\textbf{Métrique} & \textbf{Valeur} \\
\hline
Couverture de tests (Controller) & 100\% \\
Compilation sans warnings & Oui \\
Tests réussis & 9/9 \\
Architecture respectée (MVC) & Oui \\
Documentation (Javadoc) & Partielle \\
\hline
\end{tabular}
\caption{Métriques qualité du projet}
\end{table}

% ====================== CONCLUSION ======================
\newpage
\section{Conclusion}

\subsection{Bilan qualité}

Le projet \textbf{"Gestion de Bibliothèque"} respecte les standards d'assurance qualité définis :

\begin{itemize}
    \item ✅ Architecture Spring Boot MVC conforme
    \item ✅ Tests unitaires complets avec JUnit 5 et Mockito
    \item ✅ Tests d'intégration validés (contexte Spring)
    \item ✅ Taux de réussite des tests : 100\%
    \item ✅ Conventions de codage Java respectées
    \item ✅ Gestion des versions avec Git
\end{itemize}

\subsection{Perspectives d'amélioration}

\begin{enumerate}
    \item \textbf{Intégration continue} : Mise en place de GitHub Actions pour automatiser les tests
    \item \textbf{Couverture étendue} : Ajout de tests pour Repository et Service (si créés)
    \item \textbf{Tests de charge} : Évaluer les performances avec JMeter
    \item \textbf{Sécurité} : Ajout d'authentification (Spring Security)
    \item \textbf{Documentation} : Compléter les commentaires Javadoc
\end{enumerate}

\subsection{Livrables}

\begin{itemize}
    \item Code source complet (Entity, Repository, Controller, Views)
    \item Tests unitaires (LivreControllerTest.java)
    \item Rapport Maven (surefire-reports)
    \item Base de données MySQL configurée
    \item Ce document (Plan d'Assurance Qualité)
\end{itemize}

\vspace{2cm}

\begin{center}
\textit{Projet validé et conforme aux exigences qualité.}

\vspace{1cm}

\textbf{Date de validation : 17 Février 2026}
\end{center}

\end{document}
